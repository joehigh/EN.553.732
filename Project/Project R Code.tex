\section*{R Code}
\lstset{backgroundcolor=\color{light-gray}, frame=single, basicstyle = \ttfamily\small}
\begin{lstlisting}{language=R}
# This analysis attempts to predict batting average
# The goal is to compare frequentist and bayesian methods
# Data come from Lahman Baseball data archive, copyright Sean Lahman
# Source: http://www.seanlahman.com/baseball-archive/statistics/
# Authors: Kenneth Feder, Joseph High, Joseph Yu

# Load required packages -- may need to be installed
library(ggplot2)
library(brms)
library(rjags)
library(lme4)
library(plyr)
library(dplyr) 
library(magrittr)
library(coda)
library(beepr)

# Set directory to read in data
setwd("~/Documents/Johns Hopkins/Coursework/Term 9/Bayesian 
           Statistics/Project/baseballdatabank-2017/core") 

# read in player demographics and batting data

master <- read.csv("Master.csv")
batting <- read.csv("Batting.csv")

# Reset directory to working directory
setwd("~/Documents/Johns Hopkins/Coursework/Term 9/Bayesian 
Statistics/Project")

# Merge the two datasets
mlb <- right_join(master,batting,by = "playerID")
summary(mlb)

###################### Preparing for Analysis ######################
head(mlb)
filter(mlb,birthYear == 1852,birthState == "MD")
# Rearrange for easier viewability
mlb %<>% arrange(playerID,yearID)
head(mlb)
summary(mlb)

# Data cleaning for analysis

mlb.analysis <- mlb %>% select(playerID,birthYear,weight,height,
      yearID,teamID,lgID,G,AB,H) %>% # Extract relvant variables...
  filter(yearID > 1900,AB >= 100,lgID == "NL" | lgID == "AL") %>% 
  # Filter only post 1901, at least 100 AB per season, 
  # only NL and AL play
  mutate( # Add transformed variables
    age = yearID - birthYear,
    age.square = age^2,
    year = yearID - 1900,
    spline.liveball = ifelse(yearID < 1920,0,yearID - 1919), 
    # Allow slope to shift at start of liveball era using spline 
    # http://www.netshrine.com/era.html
    spline.expansion = ifelse(yearID < 1961,0,yearID - 1961), 
    # Allow slope to shift at start of expansion era
    spline.freeagency = ifelse(yearID < 1977,0,yearID - 1977), 
    # Allow slope to shift at start of free agency
    spline.steroids = ifelse(yearID < 1994,0,yearID - 1994), 
    # Allow slope to shift at start of steroid era
    spline.modern = ifelse(yearID < 2005,0,yearID - 2004), 
    # Allow slope to shift at start of modern era
    height.square = height^2,
    weight.square = weight^2,
    avg = H/AB
  ) %>% group_by(playerID) %>% # Group by player ID...
  mutate( # So we can create lagged batting average variables 
          # for each player
    avg.lag1 = lag(avg),
    avg.lag2 = lag(avg.lag1),
    season = row_number()
  ) %>% filter(season > 2) %>% # Drop the first two seasons of 
                               # a player's career
  na.omit %>%  # Drop records with missing information (there are a 
# small number of players with no height/weight info from 
# early 1900s)
  as.data.frame()

# Exploratory plot -- histogram of batting averages
plot.raw.dist <- ggplot(mlb.analysis,aes(x = avg)) +
  geom_histogram()
plot.raw.dist

# Exploratory plot -- batting trends over time
plot.raw.trends <- ggplot(mlb.analysis,aes(yearID,avg, 
          group = playerID)) + geom_line(size = .1, alpha = .5) 
plot.raw.trends

################ Fit frequentist multi-level model ################
names(mlb.analysis)
fit.freq <- lmer(avg ~ avg.lag1 + avg.lag2 + age + age.square + 
                   height + height.square + weight + weight.square + 
                   year + spline.liveball + spline.expansion + 
                   spline.freeagency + spline.steroids + 
                   spline.modern + (1|playerID),
                 data = mlb.analysis)
summary(fit.freq)

# Predictive accuracy
mlb.analysis$predict.freq <- predict(fit.freq)
plot.freqfits.vs.truth <- ggplot(mlb.analysis,aes(avg,predict.freq)) 
   + geom_point() + geom_abline(slope = 1,intercept = 0)
plot.freqfits.vs.truth

# A few example trajectories from post 1990 players
plot.example.trajectories <- mlb.analysis %>% filter(yearID > 1990) 
   %>%
  filter(playerID %in% sample(unique(playerID),12)) %>%
  ggplot() +
  geom_point(aes(yearID,avg),size = .5,alpha = .5) +
  geom_line(aes(yearID,predict.freq)) +
  facet_wrap(~playerID,scales = "free")
plot.example.trajectories

# All predicted trajectories
plot.predicted.trends <- ggplot(mlb.analysis,aes(yearID,predict.freq, 
group = playerID)) + geom_line(alpha = .3) 
plot.predicted.trends

# Overly predictions on data. Less variable, which is what we 
#would expect -- predictions do not include residual variance
plot.freq.vs.data.hist <- ggplot(mlb.analysis) + 
  geom_density(aes(predict.freq),fill = 'blue',alpha = .5) +
  geom_density(aes(avg),fill = 'red',alpha = .5)
plot.freq.vs.data.hist

################### Bayesian multi-level model  ###################

# We are using the Bayesian engine JAGS (Just Another Gibbs Sampler)
# 4 steps to every model:
# 1) specify, 2) intialize, 3) burn-in, 4) run MCMC
# NOTE -- THE MODEL ITSELF IS IN baseball_project.bug

# These are variables needed to run the model using jags
n <- nrow(mlb.analysis)
J <- length(table(mlb.analysis$playerID))
tests <- sort(sample(1:nrow(mlb.analysis),100)) 
# This is row numbers for 100 player-years we select at random to 
# make predictions on.
mlb.analysis$playerID.numeric <- as.numeric(as.factor(mlb.analysis
$playerID))

# The parameters to be returned after running the MCMC. 
# For simplicity, I only return the coefficients associated with 
# batting average
jags.parameters <- c("b1", "b2") 
# Starting values for the MCMC
jags.inits <- list(b0 = rep(0,J),b1 = 0,b2 = 0,b3 = 0,b4 = 0,g0 = 0, 
  g1 = 0, g2 = 0,g3 = 0,g4 = 0,g5 = 0, g6 = 0,g7 = 0,g8 = 0,g9 = 0, 
  g10 = 0,avg.precision = 1, 
  b0.precision = 1)
attach(mlb.analysis)
# All the data we pass to the model has to be stored in a list
jags.data <- list(
  "n" = n,
  "J" = J,
  "tests" = tests,
  "playerID" = playerID.numeric,
  "avg" = avg,
  "avg.lag1" = avg.lag1,
  "avg.lag2" = avg.lag2,
  "age" = age,
  "age.square" = age.square,
  "height" = height,
  "height.square" = height.square,
  "weight" = weight,
  "weight.square" = weight.square,
  "year" = year,
  "spline.liveball" = spline.liveball,
  "spline.expansion" = spline.expansion,
  "spline.freeagency" = spline.freeagency,
  "spline.steroids" = spline.steroids,
  "spline.modern" = spline.modern
)

# This is the key step -- we compile the model from the JAGS code, 
# the data, and the specifications we give
avg.jags.model <- jags.model(file = "baseball_project.bug", 
                             data = jags.data, 
                             inits = jags.inits, 
                             # Starting values for the MCMC
                             n.adapt = 1000, 
                             n.chains = 3 

# This is the burn in. Takes about 2 minutes
update(avg.jags.model, n.iter=1000) # burn in

# Now we actually run the MCMC. I have only 2000 iterations; 
# it takes about 5 minutes.
t <- Sys.time()
beta.samples<-coda.samples(avg.jags.model, 
variable.names = jags.parameters, n.iter = 5000,thin = 5)
Sys.time() - t
beep()

# Note the estimated coefficients are very similar to frequentist 
# estimates, which is encouraging
summary(beta.samples)
summary(fit.freq)

# Here are some convergence diagnostics
plot(beta.samples)
acfplot(beta.samples)
gelman.diag(beta.samples)



# Now let's rerun and get a posterior predictive distribution for a 
# random sample of 100 players. This is quite cumbersome, which 
# is why we did the conergence diagnostics up front
# But we want this to make a histogram to compare to our data

jags.parameters <- c("pavg") 

# This is the burn in
update(avg.jags.model, n.iter=1000) # burn in

# Now we actually run the MCMC. 
t <- Sys.time()
avg.jags.samples<-coda.samples(avg.jags.model, 
variable.names = jags.parameters, n.iter = 5000,thin = 5)
Sys.time() - t
beep()
str(avg.jags.samples)

# Make histogram of predicted values for those hundred players. 
#Overlay on true batting avg distribution.
View(prediction.matrix)
prediction.matrix <- avg.jags.samples[[1]]
all.predictions <- as.numeric(prediction.matrix)
# Not to shabby, eh?
plot.bayes.vs.data.hist <- qplot() + 
  geom_density(aes(mlb.analysis$avg),fill = 'blue',alpha = .5) +
  geom_density(aes(all.predictions),fill = 'red',alpha = .5)
plot.bayes.vs.data.hist
\end{lstlisting}
